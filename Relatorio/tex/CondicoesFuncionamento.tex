Para o correto funcionamento da nossa implementação, assume-se que o Pioneer 
será equipado com um sensor Hokuyo com aproximadamente 728 amostras de distância 
de objetos, 40 vezes por segundo, para que se possa detectar a presença de 
objetos próximos. Além de, possuir rodas de borracha de modo que o Pioneer não 
deslize ao se movimentar e tração o suficiente para permitir que o robô rode 
sobre o eixo desejado.

As condições do ambiente devem ser um chão plano para que o Pioneer possa se 
deslocar e para que o sensor de laser possa captar melhor os obstáculos, 
assumindo que os obstáculos estarão visíveis através do sensor, e que o chão 
seja não escorregadio para que o robô seja capaz de seguir fielmente a direção 
e velocidade dadas ao controlar o Pioneer.

Em relação aos obstáculos, as condições são que não pode haver um número 
demasiadamente grande de obstáculos porque, com o algoritmo implementado, o 
Pioneer não seria capaz de achar um caminho viável para desvio, pois assumiu-se 
uma distância mínima entre o robô e os obstáculos. 

Além disso, os obstáculos não podem se movimentar com uma velocidade muito 
grande perto do robô, pois, caso essa situação ocorresse, o robô poderia não ter
tempo o suficiente para calcular a nova rota e desviar a tempo do obstáculo 
podendo causar uma colisão. E, por fim, os obstáculos devem estar em uma altura
e ter o formato que possa ser detectado pelo sensor de laser e não podem ser 
capazes de absorver o raio laser.

O algoritmo utilizado VFH planeja o caminho apenas localmente e dessa forma, 
não tenta achar um caminho otimizado para o robô, implicando que, o robô pode 
ficar “preso” em situações sem fim, ou seja, ele pode andar em círculos ou 
circular entre os obstáculos para tentar o desvio e pode não voltar para o 
caminho original.
