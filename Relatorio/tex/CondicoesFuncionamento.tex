Para o correto funcionamento da atual implementação, assume-se que o Pioneer 
será equipado com um sensor laser.

As condições do ambiente devem ser um terreno plano e com espaço suficiente para que o Pioneer possa se 
deslocar. Além disso, o terreno não deve ser escorregadio, para evitar colisões inesperadas. 
 
Os obstáculos não podem se movimentar com uma velocidade elevada quando próximos ao robô, pois, caso contrário, o robô poderá não ter tempo suficiente para calcular a nova rota, causando uma colisão com o obstáculo. 
Por fim, os obstáculos devem possuir textura e coloração refletora e altura mínima, de forma que o laser consiga detectá-los.

O algoritmo VFH planeja o caminho apenas localmente e, dessa forma, 
não tenta achar um caminho otimizado para o robô, implicando que este pode 
ficar “preso” indefinidamente em algumas situações. Neste caso, ele geralmente andará em círculos ou 
circulará em torno de um conjunto de obstáculos próximos.
