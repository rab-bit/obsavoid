A partir desse estudo, foi possível implementar, simular e colocar 
em prática a movimentação do robô sem colidir com obstáculos. O VFH
se mostrou uma alternativa válida para esse desafio; com apenas o 
comando de direção e a navegação sensorial, o robô mostrou uma autonomia
para explorar um espaço bidimensional sem maiores problemas. Mesmo quando
submetido ao surgimento abrupto de obstáculos em sua frente, esse mostrou-se
capaz de tomar decisões rápidas e eficazes de forma que fosse evitada a
colisão e também o que tange o replanejamento de trajetória. Parte disso deve-se à
boa performance do algoritmo, capaz de realizar cálculos em tempo real, mesmo
com a limitação de processamento do embarcado; é também razoável levar em
consideração a taxa de atualização do sensor, que providencia uma ampla 
amostra de dados, possibilitando trabalhar sempre com o ambiente mais atual.
