Os robôs estão cada vez mais autônomos, e quando se trata de
exploração de um ambiente, uma importante tarefa que ele deve estar
apto a realizar é a prevenção de obstáculos em seu caminho. Essa
tarefa é dinâmica, ou seja, não é necessário ter um conhecimento
prévio do ambiente em que o robô se encontra, contando apenas com
algoritmos que trabalham com dados recebidos pelos
sensores. Tipicamente, o desvio de obstáculos recebe instruções de
outro módulo (geralmente o planejador de rota) que indica qual direção
ele deve seguir, e assim, decide o movimento que o motor deve realizar
baseado nos dados que os sensores fornecem, realizando, portanto, uma
interface entre o módulo de decisão de caminho e o motor do robô.
Atualmente, existem diversos algoritmos que propõem a resolução do
desvio de obstáculos. O utilizado neste projeto foi o Vector Field
Histogram(VFH).